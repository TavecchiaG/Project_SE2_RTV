\documentclass[10pt, a4paper,titlepage]{article}

\usepackage{graphicx}
\usepackage{listings}
\usepackage[normalem]{ulem}
\usepackage{caption}
\usepackage[bottom]{footmisc}

\begin{document}
\begin{titlepage}
\title{Integration Testing Document \\Version 1.0}
\author{Emanuele Ricciardelli (mat. 875221) \and Giorgio Tavecchia (mat. 874716) \and Francesco Vetr\'o (mat. 877593)}
\begin{figure}
\begin{center}
\includegraphics{/home/francesco/git/Project_SE2_RTV/Integration_Testing/TEST_images/logopolimi.png}
\caption{Politecnico di Milano}
\label{fig:logo}
\end{center}
\end{figure}
\maketitle
\end{titlepage}
\tableofcontents
\pagebreak
\section{Introduction}
\subsection{Purpose}
This document represents the Integration Testing Plan Document for PowerEnJoy service.
The purpose of the testing is to verify and ensure that the system operation is correct and complies with all the requirements expressed in the previous documents (RASD and DD). All components of the system will have to comply with the constraints and functionalities for which they were designed, without showing unexpected behavior.
The purpose of this document resides therefore in specifying the methods and the time needed to complete a full testing, both from the point of view of the individual components, both in their interactions.
In order to provide a detailed description of the future testing, in the following paragraphs it will be defined:
\begin{itemize}
\item A list of the subsystems and their subcomponents of the PowerEnJoy system-to-be that will be involved in the testing activity;
\item The criteria that must be met by the project status before integration testing of the outlined elements may begin;
\item A description of the integration testing approach and the rationale behind the specific choose;
\item The sequence of component and function integration;
\item A description of individual steps and test description: for each step of integration, we will define a test design with related expected results for each test;
\item A list of tools, test equipment and test data required.
\end{itemize}•
\subsection{Definitions and Abbreviations}
\begin{itemize}
\item RASD: Requirement Analysis and Specification Document;
\item DD: Design Document;
\end{itemize}•
\emph{For all other definitions, please refer to previous documents listed in the next paragraph}
\subsection{Reference Documents}
\begin{itemize}
\item Specification document: Assignments AA 2016-2017.pdf
\item RASD Version 1.1
\item DD Version 1.1
\item Integration testing example document.pdf
\end{itemize}•
\section{Integration strategy}
\subsection{Entry criteria}
With reference to the past documents and in particular to the design document, it is necessary to specify how the individual components that make up the system require to achieve a given percentage of completion with respect to the features for which they were designed.
Especially, with respect to the High Level Component view of the Design Document:
\begin{itemize}
\item 100\% for the DBMS component: in order to allow a correct testing not only of the database structure itself, but also of the various components of the system that base their operation on access to the database;
\item 60\% of the Client components: many of the functions covered by the client involve the GUI and have no priority over interaction with the system functionality;
\item 90\% of the Application Server;
\item 90\% of the Car Sensor Subsystem: in order to ensure a correct testing of the application server, given the close interaction.
\end{itemize}•
These criteria must be met before the start of the Integration testing. Different percentages want to reflect the order of integration, placed its focus on certain components, and otherwise in the background some secondary aspects such as the graphical user interface on the client.
\subsection{Elements to be integrated}
In order to specify the elements to be integrated, we refer not only to High Level Component view contained in the DD, but also to the Component view that allows us to identify components at a lower level and the relationships between them .
As specified in the Design Document, the components of our system have been designed in such a way as to ensure a high level of modularity, by separating the different functionality into distinct components, thus also facilitating testing.
For this reason, right from the Component view it is possible to note how different components are strongly dependent on one another in order to achieve more complex functionality of the system-to-be.
The recognition of these complex functionality allows us to split our system into a number of subsystems, addressed respectively to the access management and registration system (\emph{Account Subsystem}), management of payments (\emph{Payment subsystem}), reservation management and everything that concerns the sphere of rental (\emph{Reservation subsystem}) and finally management of all the sensors and the exchange of information between cars and system and everything related to the maintenance (\emph{Car / Sensors subsystem}).
In particular, the functions relating to the \emph{Reservation subsystem} will involve the integration of:
\begin{itemize}
\item Reservation Controller;
\item Ride Controller (integrated with Google Maps API);
\item Notification Helper (with related gateway);
\item Communication Manager;
\item Request Manager;
\item Data Access Manager.
\end{itemize}•
With respect to the \emph{Account subsystem}, it is concentrated only in the \emph{Login / Registration Controller} that later, through the \emph{Communication Manager}, will interface with the other components of the PowerEnJoy system. In detail:
\begin{itemize}
\item Login/Registration Controller;
\item Communication Manager;
\item Request Manager;
\item Notification Helper (with related gateway);
\item Data Access Manager.
\end{itemize}•
The same reasoning can be made for the \emph{Payment subsystem} and the respective \emph{Payment Controller}. In detail:
\begin{itemize}
\item Request Manager;
\item Payment Controller;
\item Communication Manager;
\item Data Access Manager;
\item Payment Gateway;
\end{itemize}
Finally, the \emph{Car / Sensors subsystem} in order to perform all its functionalities will need to integrate:
\begin{itemize}
\item Cars Controller;
\item Communication Manager;
\item Collector Data (with related gateway);
\item Maintenance Gateway;
\item Request Manager;
\item Data Access Manager;
\end{itemize}•
It is important to underline that the components \emph{Notification Helper} \emph{Request Manager} and \emph{Communication Manager} are significant for all subsystems and therefore their integration. This means that if on the one hand their presence ensures greater flexibility in the management of functions, on the other hand emphasizes the importance of their proper functioning in order to ensure the fulfillment of all the functionality of the overall system.
Given the above definitions of subsystems, our integration testing will proceed initially considering a single subsystem at a time, following finally with the integration of the same until the establishment of the system in its entirety.
\subsection{Integration testing strategy}
The approach suggested in the process of integration testing is bottom-up.
This choice is based on considerations relating to the features covered by our system: in particular, as has been documented in the DD, it has been chosen to decompose the system into a series of components, each with features related to a particular field. The functionalities that the system will perform are derived from joint work of these components. From this observation it can be seen that the choice of a bottom-up reflects the needs of the testing related to our system.
In this way we can favour the ability to identify problems at an early stage of development of the components and, in case, to react quickly. 
Later, after having tested the elementary interactions between components and subsystems, we can proceed with the integration to a higher level.
It will therefore be necessary to develop a set of drivers to drive the testing, but given the simultaneous development of the overall system, this will prove to be quite natural.
Once the system will be completely formed, we shall provide the implementation of a series of System Testing in order to verify not only the achievement of the functional requirements, as a whole, but also of the non-functional aspects such as the loading capacity as a function of the expected values.
\subsection{Sequence of component / function integration}
In this section we will provide a description of the order of integration testing of components and subsystems of PowerEnJoy system-to-be.
\subsubsection{Software integration sequence}
Given the previous subdivision into subsystems and starting from the \emph{Reservation subsystem}, we now define the sequence in which the components will be integrated for each subsystem and then the sequence related to the integration of the subsystems themselves.
\paragraph{Reservation subsystem}\mbox{}\\
Since many features covered by this system need an interaction with the database (performing queries or updating tables), the very first components to be integrated are the \textbf{Data Access Manager} and the \textbf{DBMS}.
\begin{figure}[h]
\begin{center}
\includegraphics[width=\textwidth, height=0.2\textheight, keepaspectratio]{/home/francesco/git/Project_SE2_RTV/Integration_Testing/TEST_images/Integration_Sequence1.png}
\caption{Integration sequence - 1 - Data Access Manager, DBMS}
\label{fig:int_seq1}
\end{center}•
\end{figure}
As will later be made clear, this first integration will also be required for other subsystems, but will not be repeated if it is unnecessary; otherwise it will be repeated the possible variants of integration between these first components and the surrounding ones.
With respect to the bottom-up approach, the next integration step will cover the correct communication between the database and the \textbf{Communication Manager}.
\begin{figure}[h]
\begin{center}
\includegraphics[width=\textwidth, height=0.3\textheight, keepaspectratio]{/home/francesco/git/Project_SE2_RTV/Integration_Testing/TEST_images/Integration_Sequence2.png}
\caption{Integration sequence - 2 - Communication Manager \& above}
\label{fig:int_seq2}
\end{center}•
\end{figure}
We can now proceed to a higher level of integration considering the \textbf{Ride Controller}:
this component will be integrated with the API provided by the services of google in order to verify the correct use within their own methods, with the "Notification Helper" in order to verify the correct sending of notifications generated to users, and finally with \textbf{Communication Manager} which, as described in the DD, is designed to allow (in a flexible and transparent way) the exchange of information between internal components of the system, to verify proper access to the database.
\begin{figure}[h]
\begin{center}
\includegraphics[width=\textwidth, height=0.4\textheight, keepaspectratio]{/home/francesco/git/Project_SE2_RTV/Integration_Testing/TEST_images/Integration_Sequence3.png}
\caption{Integration sequence - 3 - Ride Controller, Notification Helper, Google APIs \& above}
\label{fig:int_seq3}
\end{center}•
\end{figure}
At the same level of integration we will introduce the \textbf{Reservation Controller} integrated with \textbf{Communication Manager} and \textbf{Notification Helper} for the same reasons mentioned above.
\begin{figure}[h]
\begin{center}
\includegraphics[width=\textwidth, height=0.4\textheight, keepaspectratio]{/home/francesco/git/Project_SE2_RTV/Integration_Testing/TEST_images/Integration_Sequence4.png}
\caption{Integration sequence - 4 - Reservation Controller \& above}
\label{fig:int_seq4}
\end{center}•
\end{figure}
\clearpage
Since different functions performed by the controller such as \textbf{Reservation Controller} and \textbf{Ride Controller} are performed by request from external clients, and given that the same demands are made through the component \textbf{Request Manager}, the next components to be integrated will be the following.
\begin{figure}[h]
\begin{center}
\includegraphics[width=\textwidth, height=0.5\textheight, keepaspectratio]{/home/francesco/git/Project_SE2_RTV/Integration_Testing/TEST_images/Integration_Sequence5.png}
\caption{Integration sequence - 5 - Request Manager \& above}
\label{fig:int_seq5}
\end{center}•
\end{figure}
\clearpage
This will make it possible to test the correct routing of requests towards the selected components and the exchange of information between them.
Finally we integrate the Client, to test that all of the functionalities of this subsystem work properly.
\begin{figure}[h]
\begin{center}
\includegraphics[width=\textwidth, height=0.5\textheight, keepaspectratio]{/home/francesco/git/Project_SE2_RTV/Integration_Testing/TEST_images/Integration_Sequence6.png}
\caption{Integration sequence - 6 - Client \& above}
\label{fig:int_seq6}
\end{center}•
\end{figure}
\clearpage
\paragraph{Account subsystem}\mbox{}\\
The starting point relative to the sequence of integration of this subsystem should be regarding the access to the database, but, as previously expressed, since it was done for the previous subsystem, it is not required to do it again. It is now shown for the sake of completeness to the reader.
\begin{figure}[h]
\begin{center}
\includegraphics[width=\textwidth, height=0.2\textheight, keepaspectratio]{/home/francesco/git/Project_SE2_RTV/Integration_Testing/TEST_images/Integration_Sequence1.png}
\caption{Integration sequence - 1 - Data Access Manager, DBMS}
\label{fig:int_seq1_1}
\end{center}•
\end{figure}
\begin{figure}[h]
\begin{center}
\includegraphics[width=\textwidth, height=0.3\textheight, keepaspectratio]{/home/francesco/git/Project_SE2_RTV/Integration_Testing/TEST_images/Integration_Sequence2.png}
\caption{Integration sequence - 2 - Communication Manager \& above}
\label{fig:int_seq2_1}
\end{center}•
\end{figure}
\pagebreak
We follow now with the integration between the \textbf{Login / Registration Controller} and \textbf{Notification Helper} to verify, as done in the previous subsystem, the correct forwarding of notifications generated and, at the same time, integrating  the \textbf{Login / Registration Controller} with \textbf{Communication Manager} and \textbf{Data Access Manager} in order to test the proper functioning of methods that require database access.
\begin{figure}[h]
\begin{center}
\includegraphics[width=\textwidth, height=0.4\textheight, keepaspectratio]{/home/francesco/git/Project_SE2_RTV/Integration_Testing/TEST_images/Integration_Sequence7.png}
\caption{Integration sequence - 7 - Login/Registration Controller, Notification Helper \& above}
\label{fig:int_seq7}
\end{center}•
\end{figure}
Going up to a higher level of integration, we introduce the \textbf{Request Manager} in order to test the correct routing of requests coming from the outside to the component suited to the management of the system accounts.
\begin{figure}[h]
\begin{center}
\includegraphics[width=\textwidth, height=0.4\textheight, keepaspectratio]{/home/francesco/git/Project_SE2_RTV/Integration_Testing/TEST_images/Integration_Sequence8.png}
\caption{Integration sequence - 8 - Request Manager \& above}
\label{fig:int_seq8}
\end{center}•
\end{figure}
And again we terminate this sequence of integration introducing the Clients to test that all of the functionalities of this subsystem work properly.
\begin{figure}[h]
\begin{center}
\includegraphics[width=\textwidth, height=0.4\textheight, keepaspectratio]{/home/francesco/git/Project_SE2_RTV/Integration_Testing/TEST_images/Integration_Sequence9.png}
\caption{Integration sequence - 9 - Client \& above}
\label{fig:int_seq9}
\end{center}•
\end{figure}
Thus completing the integration of this subsystem.
\clearpage
\paragraph{Payment subsystem}\mbox{}\\
We now proceed in the integration process Considering the \emph{Payment Subsystem}.
As before, we start integrating the components related to the access to the database.
\begin{figure}[h]
\begin{center}
\includegraphics[width=\textwidth, height=0.2\textheight, keepaspectratio]{/home/francesco/git/Project_SE2_RTV/Integration_Testing/TEST_images/Integration_Sequence1.png}
\caption{Integration sequence - 1 - Data Access Manager, DBMS}
\label{fig:int_seq1_2}
\end{center}•
\end{figure}
\begin{figure}[h]
\begin{center}
\includegraphics[width=\textwidth, height=0.3\textheight, keepaspectratio]{/home/francesco/git/Project_SE2_RTV/Integration_Testing/TEST_images/Integration_Sequence2.png}
\caption{Integration sequence - 2 - Communication Manager \& above}
\label{fig:int_seq2_2}
\end{center}•
\end{figure}
\pagebreak
The first components that need to be integrated are the \textbf{Payment Controller} and the \textbf{Payment Gateway}, this to verify that the results relating to payment functions are correctly sent to the external payment companies, and the integration between \textbf{Payment Controller} and \textbf{Communication Manager} to test the communication with the database.
\begin{figure}[h]
\begin{center}
\includegraphics[width=\textwidth, height=0.3\textheight, keepaspectratio]{/home/francesco/git/Project_SE2_RTV/Integration_Testing/TEST_images/Integration_Sequence10.png}
\caption{Integration sequence - 10 - Payment Controller, Payment Gateway \& above}
\label{fig:int_seq10}
\end{center}•
\end{figure}
Followed by the integration with the \textbf{Request Manager} to verify the correct forwarding of information requests to the subsystem.
\begin{figure}[h]
\begin{center}
\includegraphics[width=\textwidth, height=0.4\textheight, keepaspectratio]{/home/francesco/git/Project_SE2_RTV/Integration_Testing/TEST_images/Integration_Sequence11.png}
\caption{Integration sequence - 11 - Request Manager \& above}
\label{fig:int_seq11}
\end{center}•
\end{figure}
As before we terminate this sequence of integration testing the connection with the Client.
\begin{figure}[h]
\begin{center}
\includegraphics[width=\textwidth, height=0.4\textheight, keepaspectratio]{/home/francesco/git/Project_SE2_RTV/Integration_Testing/TEST_images/Integration_Sequence12.png}
\caption{Integration sequence - 12 - Client Payment System \& above}
\label{fig:int_seq12}
\end{center}•
\end{figure}
\clearpage
\paragraph{Car / Sensors subsystem}\mbox{}\\
The last subsystem to be included in the sequence of integration is the \emph{Car / Sensors subsystem} that will handle all exchanges of information and operations to be performed on vehicles and everything related to the maintenance.
The starting point is still represented by the access to database.
\begin{figure}[h]
\begin{center}
\includegraphics[width=\textwidth, height=0.2\textheight, keepaspectratio]{/home/francesco/git/Project_SE2_RTV/Integration_Testing/TEST_images/Integration_Sequence1.png}
\caption{Integration sequence - 1 - Data Access Manager, DBMS}
\label{fig:int_seq1_3}
\end{center}•
\end{figure}
\begin{figure}[h]
\begin{center}
\includegraphics[width=\textwidth, height=0.3\textheight, keepaspectratio]{/home/francesco/git/Project_SE2_RTV/Integration_Testing/TEST_images/Integration_Sequence2.png}
\caption{Integration sequence - 2 - Communication Manager \& above}
\label{fig:int_seq2_3}
\end{center}•
\end{figure}
As first components to be integrated, we choose \textbf{Cars Controller} and \textbf{Data Collector}: this choice derives from the importance that covers the proper exchange of information between system and machine, upon which are based a considerable number of other functions related to the \textbf{Cars Controller} itself. In addition we integrate our principal component with the \textbf{Maintenance Gateway}  to test the proper sending of information from the system to the maintenance company and with the \textbf{Communication Manager} to verify the correct access with the database.
\begin{figure}[h]
\begin{center}
\includegraphics[width=\textwidth, height=0.4\textheight, keepaspectratio]{/home/francesco/git/Project_SE2_RTV/Integration_Testing/TEST_images/Integration_Sequence13.png}
\caption{Integration sequence - 13 - Cars Controller, Collector Data, Maintenance Gateway \& above}
\label{fig:int_seq13}
\end{center}•
\end{figure}
Finally the sequence of integration will end integrating the \textbf{Request Manager} and so to verify the correct implementation of functionalities such as reporting a fault to a machine or access by the maintenance to the list of reported cars.
\begin{figure}[h]
\begin{center}
\includegraphics[width=\textwidth, height=0.4\textheight, keepaspectratio]{/home/francesco/git/Project_SE2_RTV/Integration_Testing/TEST_images/Integration_Sequence14.png}
\caption{Integration sequence - 14 - Request Manager \& above}
\label{fig:int_seq14}
\end{center}•
\end{figure}
As last integration, like in other subsystems, we verify and test the integration with the clients.
\begin{figure}[h]
\begin{center}
\includegraphics[width=\textwidth, height=0.4\textheight, keepaspectratio]{/home/francesco/git/Project_SE2_RTV/Integration_Testing/TEST_images/Integration_Sequence15.png}
\caption{Integration sequence - 15 - Clients \& above}
\label{fig:int_seq15}
\end{center}•
\end{figure}
\clearpage
\emph{\underline{Note to the reader}}: In previous integration sequences it was performed a choice concerning the gateway present in the component view of the DD: we decide to show explicitly only the payment and maintenance gateway. This is because the integration of the omitted gateway is to consider joint with the components to which they are directly linked: For example, the \textbf{Sensor Gateway} is considered as a joint to the component \textbf{Data Collector} and the gateway relative to the notifications joint with the \textbf{Notification Helper}.
\pagebreak
\subsubsection{Subsystem integration sequence}
\end{document}